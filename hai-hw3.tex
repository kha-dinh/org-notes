% Created 2021-04-12 Mon 23:42
% Intended LaTeX compiler: pdflatex
\documentclass[11pt]{article}
\usepackage[utf8]{inputenc}
\usepackage[T1]{fontenc}
\usepackage{graphicx}
\usepackage{grffile}
\usepackage{longtable}
\usepackage{wrapfig}
\usepackage{rotating}
\usepackage[normalem]{ulem}
\usepackage{amsmath}
\usepackage{textcomp}
\usepackage{amssymb}
\usepackage{capt-of}
\usepackage{hyperref}
\usepackage[margin=1in]{geometry}
\author{DINH DUY KHA - 2019712308}
\date{\today}
\title{Implementing an online pharmaceutical service using design science research - Paper Sumary}
\hypersetup{
 pdfauthor={DINH DUY KHA - 2019712308},
 pdftitle={Implementing an online pharmaceutical service using design science research - Paper Sumary},
 pdfkeywords={},
 pdfsubject={},
 pdfcreator={Emacs 28.0.50 (Org mode 9.5)}, 
 pdflang={English}}
\begin{document}

\maketitle
\tableofcontents

\textbf{Summary}

In this paper, the authors apply Design Science Research in implementing \textbf{ePharmacare}, an online pharmaceutical service. Different activities of Design Science Research Methodology (DSRM) were carefully explained and evaluated. Finally, they report and analyze the results.
By applying DSRM into the implementation, they achieved positive results.Overall, pharmacists saves \(50\%\) of their service time while patients value using the online services more. The authors also identify a limitation in such system, where the pharmacists are not familiar with managing an information system.

DSRM was chosen because of its usefulness in connecting research and professional practices. Overall, six activities of DSRM as established in the literature is used. The first activity is \textbf{identify problem \& motivate}, in which they perform a scenarios planning exercise. The exercise consider two objectives: to analyze the posible evolution of phramacy and identify the possible uncertainties. An online survey and a following observeytional study were performed to sastisfy the defined objectives.
The second activity is \textbf{Defining objectives of a solution}. this activity aims to identify the services required by a patients when they use pharmaceutical services. The authors performs 50 semi-structured interviews that focus on customer service to gather opinions of the current service.
In third activity, \textbf{Design \& development} is performed using  Dader methodology and \textbf{Service Experiment Bluprint}. Agile methodology is used in developing the software of the online service, in which a new version is published every week to be evaluated by the users.
The fourth activity is \textbf{demonstration}, which tests the services in two settings over 9 months, one includes consultations from pharmacists, one doesn't.
The fifth activity is the \textbf{evaluation} of the system, which takes place eight months after the last acivity. It includes tess with 4 pharmacists and 3 patients using eye-tracking glasses to study which platform features were most focused on. the tests also include interview to get more information from the users.
The final activity is \textbf{Communication}, which aim to convey the results through different mediums, from oral presentation to paper publishcations into conferences and journals.

\textbf{Critique}

Overall, this is an interesting paper which tries to combine different disiplines to achieves the best possible results for a pharmaceutical online service. The platform produced is of high quality and usability with careful design decicions based on Design Science Research Methods.
Finally, it helps to prevent, manage and control chronic medications efficently, which has been shown to heavity burden the healcare service.
\end{document}
